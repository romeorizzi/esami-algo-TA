% Template per generare

\documentclass[a4paper,11pt]{article}
\usepackage{lmodern}
\renewcommand*\familydefault{\sfdefault}
\usepackage{sfmath}
\usepackage[utf8]{inputenc}
\usepackage[T1]{fontenc}
\usepackage[italian]{babel}
\usepackage{indentfirst}
\usepackage{graphicx}
\usepackage{tikz}
\newcommand*\circled[1]{\tikz[baseline=(char.base)]{
    \node[shape=circle,draw,inner sep=2pt] (char) {#1};}}
\usepackage{enumitem}
% \usepackage[group-separator={\,}]{siunitx}
\usepackage[left=2cm, right=2cm, bottom=3cm]{geometry}
\frenchspacing

\newcommand{\num}[1]{#1}

% Macro varie...
\newcommand{\file}[1]{\texttt{#1}}
\renewcommand{\arraystretch}{1.3}
\newcommand{\esempio}[2]{
  \noindent\begin{minipage}{\textwidth}
    \begin{tabular}{|p{11cm}|p{5cm}|}
      \hline
      \textbf{File \file{input (da stdin)}} & \textbf{File \file{output (su stdout)}}\\
      \hline
      \tt \small #1 &
      \tt \small #2 \\
      \hline
    \end{tabular}
  \end{minipage}
}

% Dati del task
\newcommand{\gara}{.}
\newcommand{\nome}{Biancaneve al riordino dei nanetti}
\newcommand{\nomebreve}{biancaneve}

\begin{document}
  
  
  % Intestazione
  \noindent{\Large \gara}
  \vspace{0.5cm}
  
  \noindent{\Huge \textbf \nome~(\texttt{\nomebreve})}
  \vspace{0.2cm}\\

  \begin{flushright}
      Questo problema \`e una rielaborazione del task SNJEGULJICA, COCI 2012/2013, $7^{th}$ round.
  \end{flushright}
  
  % Descrizione del task
  \section*{Descrizione del problema}
    
  \noindent
  Biancaneve \`e dedita al riordino dei suoi $n$ nanetti da giardino.
  I nanetti sono disposti in riga, nelle posizioni da $1$ ad $n$,
  e questo le va bene.
  Ci\`o che non le piace \`e che i nanetti sono tutti di altezza diversa
  (ciascuna rappresentata con un diverso naturale da $1$ ad $n$)
  e poi c'\`e che la fila non le risulta bene ordinata.
  Procede pertanto al riordino della fila impartendo ordini di due possibili tipologie:
  \begin{description}
     \item[1 $p_1$ $p_2$] a seguito di questo comando i nani nelle posizioni $p_1$ e $p_2$ della fila si scambiano di posto; tutti gli altri nani mantengono la loro posizione. In sintesi: date due posizioni, scambiare di posto i nanetti che le occupano. 
     \item[2 $h_1$ $h_2$] questo comando non altera la disposizione dei nani,
       ma richiede di stabilire se i nani di altezze $h_1, h_1+1, \ldots, h_2$ sono raggruppati, ossia occupano un intervallo di posizioni contigue (anche se non necessariamente in quell'ordine).
        In sintesi: dato un range di altezze, stabilire se i nanetti la cui altezza ricada in tale range occupino un intervallo di posizioni contigue.
  \end{description}

  Assistere i nani ad espletare gli ordini di Biancaneve e, contestualmente, a rispondere alle sue domande. 


\section*{Dati di input}

L'input deve avvenire da stdin.
Come input leggete $2+m$ righe da stdin.

La prima riga contiene gli interi positivi $n$ ed $m$
($2\leq n \leq 100\,000$, $2\leq m \leq 100\,000$),
il numero di nani ed il numero totale di richieste avanzate da Biancaneve.

La riga seguente contiene una permutazione dei numeri da $1$ ad $n$
intesa a specificare la disposizione iniziale dei nanetti:
per ogni posizione si precisa chi la occupi, dove le altezze identificano univocamente i nani. 

Ciascuna delle seguenti $m$ righe specifica un comando di Biancaneve,
come nell'odine in cui questi vengono impartiti.
Ciascuno di questi ordini viene specificato tramite una terna di numeri,
dove il primo numero, che pu\`o essere $1$ oppure $2$, specifica la tipologia del comando, ed i successivi due numeri ne costituiscono i parametri come illustrato sopra. 


\section*{Dati di output}
  L'input deve avvenire su stdout.
  Come output, dovete stampare su stdout una riga per ogni richiesta di tipo $2$
  avanzata da Biancaneve. La riga contiene ``YES'' oppure ``NO'' come risposta
  alla richiesta che le corrisponde.
  
\section*{Subtask}
  Un totale di $100$ punti \`e ripartito sui seguenti subtask.
  \begin{itemize}
    \item \textbf{Subtask 1 [0 punti]:} i casi di esempio.
    \item \textbf{Subtask 2 [20 punti]:} i nanetti non vengono mai spostati.
    \item \textbf{Subtask 3 [20 punti]:} massimo 100 nanetti e massimo 1000 query.
    \item \textbf{Subtask 4 [20 punti]:} massimo 100 nanetti e massimo 1000 spostamenti.
    \item \textbf{Subtask 5 [20 punti]:} massimo 100\,000 nanetti, massimo 100\,000 spostamenti, e massimo 100\,000 query.
    \item \textbf{Subtask 6 [2*10=20 punti]:} le 10 istanze originali a COCI danno $2$ punti ciascuna. A quella gara era dato sapere che in almeno metà delle istanze valeva che $h_2-h_1\leq 50$ per ogni richiesta di tipo~2. I nanetti erano massimo 200\,000 nanetti e le operazioni prescritte (query o spostamenti) erano massimo 200\,000 in tutto. 
  \end{itemize}

% Esempi
\section*{Esempio di input/output}
\setlength{\tabcolsep}{6pt}
\esempio{
5 3

2 4 1 3 5

2 2 5

1 3 1

2 2 5
}{
NO

YES
}
  
\esempio{
7 7

4 7 3 5 1 2 6

2 1 7

1 3 7

2 4 6

2 4 7

2 1 4

1 1 4

2 1 4
}{
YES

NO

YES

NO

YES}
  

\end{document}
