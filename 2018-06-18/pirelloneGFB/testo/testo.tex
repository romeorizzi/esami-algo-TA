\documentclass[a4paper,11pt]{article}
\usepackage{lmodern}
\renewcommand*\familydefault{\sfdefault}
\usepackage{sfmath}
\usepackage[utf8]{inputenc}
\usepackage[T1]{fontenc}
\usepackage[italian]{babel}
\usepackage{indentfirst}
\usepackage{graphicx}
\usepackage{tikz}
\usepackage{wrapfig}
\newcommand*\circled[1]{\tikz[baseline=(char.base)]{
		\node[shape=circle,draw,inner sep=2pt] (char) {#1};}}
\usepackage{enumitem}
% \usepackage[group-separator={\,}]{siunitx}
\usepackage[left=2cm, right=2cm, bottom=3cm]{geometry}
\frenchspacing

\newcommand{\num}[1]{#1}

% Macro varie...
\newcommand{\file}[1]{\texttt{#1}}
\renewcommand{\arraystretch}{1.3}
\newcommand{\esempio}[2]{
\noindent\begin{minipage}{\textwidth}
\begin{tabular}{|p{11cm}|p{5cm}|}
	\hline
	\textbf{File \file{input.txt}} & \textbf{File \file{output.txt}}\\
	\hline
	\tt \small #1 &
	\tt \small #2 \\
	\hline
\end{tabular}
\end{minipage}
}

\newcommand{\sezionetesto}[1]{
    \section*{#1}
}

\newcommand{\gara}{esame di Algoritmi 2018-06-18, Verona}

%%%%% I seguenti campi verranno sovrascritti dall'\include{nomebreve} %%%%%
\newcommand{\nomebreve}{pirelloneGFB}
\newcommand{\titolo}{Luci della Ribalta al Pirellone}

% Modificare a proprio piacimento:
\newcommand{\introduzione}{
    \noindent{\Large \gara{}}

    \vspace{0.5cm}
    \noindent{\Huge \textbf \titolo{}~(\texttt{\nomebreve{}})}
    \vspace{0.2cm}\\
}

\begin{document}

\renewcommand{\nomebreve}{pirelloneGFB}
\renewcommand{\titolo}{Luci della Ribalta al Pirellone}

\introduzione{}

In questo revival, il Pirellone è una matrice di $M$ righe ed $N$ colonne,
ed ogni cella di questa matrice è un numero naturale nell'intervallo $[0,B)$.
(Quindi, nelle istanze con $B=2$, la matrice è rimasta binaria come nel problema che abbiamo già conosciuto, mentre, quando $B>2$,
per le nuove lampade alogene negli uffici sono ora possibili più livelli di intensità invece che solo acceso/spento).
  
Il custode ha a disposizione $M$ interruttori di riga ed $N$
interruttori di colonna.  Ogni volta che il custode agisce
su un interruttore di riga o colonna ciascuna cella $P_{i,j}$ di quella riga o colonna subisce la trasformazione:
\[
   P_{i,j} \mapsto (P_{i,j}+1) .mod. B.
\]
Ad esempio, nel caso di $B=2$ (matrice $0/1$, ossia acceso/spento)
tutte le luci accese di quella riga o colonna si spengono ma, allo stesso tempo, quelle spente si accendono!

Aiuta il custode a decidere quante volte agire su ciascuno degli $M+N$
interruttori in modo da portare allo stato di spento ($0$) tutte le luci dalle finestre del Pirellone (rappresentate dalle celle della matrice). Data la configurazione iniziale di luci, il custode
deve verificare se sia possibile spegnere le luci con gli interruttori
di riga e colonna e, in tal caso, deve specificare anche quante volte agire su ciascun interruttore.


\sezionetesto{Dati di input}
  
Il file \verb'input.txt' contiene nella prima riga gli interi
$M$, $N$ e $B$ nell'ordine e separati da spazi.

Ognuna delle successive $M$ righe contiene una sequenza di $N$ valori naturali presi dall'intervallo $[0,B)$, e separati da uno spazio.
La sequenza contenuta nell'$i$-esima di tali righe rappresenta lo stato
delle luci nell'$i$-esima riga (piano) del Pirellone. In
particolare, il $j$-esimo valore della riga indica lo stato della luce nel $j$-esimo ufficio a quel piano del palazzo.
  Poichè $B \leq 10$ in tutte le istanze, questi numeri sono tutti di una sola cifra. La riga non contiene altri caratteri.



\sezionetesto{Dati di output}
  
Il file \verb'output.txt' deve contenere due linee per indicare
su quali interruttori deve agire il custode.

La prima linea contiene una sequenza di $M$ numeri naturali nell'intervallo $[0,B)$ separati da uno spazio, per rappresentare quante volte il custode deve agire su ciascuno degli $M$ interruttori di riga.

Analogamente, la seconda linea contiene una sequenza di $N$
numeri naturali nell'intervallo $[0,B)$ separati da uno spazio.
Il  $j$-esimo valore della sequenza indica quante volte
il custode deve agire sull'interruttore che agisce sulla $j$-esima colonna incrementando di uno tutte le intensità tranne per le luci di intensità massima che vengono invece spente.

Nel caso in cui non sia possibile spegnere tutte le luci del Pirellone
con gli interruttori speciali, tutti gli $M+N$ valori delle due linee in
\texttt{output.txt} devono essere settati a $0$.

  \section*{Assunzioni}
  \begin{itemize}
  
    \item $ 1 \leq M, N \leq 5000$
    \item  inizialmente il Pirellone ha almeno una luce accesa
  \end{itemize}


% Esempi
\sezionetesto{Esempio di input/output}
\esempio{
5 5 2

1 0 1 1 0

0 1 0 0 1

1 0 1 1 0

0 1 0 0 1

0 1 0 0 1
}{
1 0 1 0 0

0 1 0 0 1
}

\esempio{
4 5 2

1 0 1 1 0

0 1 1 0 1

1 0 1 1 0

0 1 0 0 1
}
{
0 0 0 0

0 0 0 0 0
}

\esempio{
8 10 3

1 1 1 0 0 0 2 0 0 0

2 2 2 1 1 1 0 1 1 1

0 0 0 2 2 2 1 2 2 2

1 1 1 0 0 0 2 0 0 0

1 1 1 0 0 0 2 0 0 0

2 2 2 1 1 1 0 1 1 1

0 0 0 2 2 2 1 2 2 2

1 1 1 0 0 0 2 0 0 0
}
{
0 2 1 0 0 2 1 0

2 2 2 0 0 0 1 0 0 0
}


  \section*{Subtask}
  \begin{itemize}
    \item \textbf{Subtask 0 [0 punti]:} i due esempi del testo.
    \item \textbf{Subtask 1 [3 punti]:} $B=2,3$, $M = 1$.
    \item \textbf{Subtask 2 [10 punti]:} $B=2,3$, $M,N \leq 10$.
    \item \textbf{Subtask 3 [12 punti]:} $B=2$, $M,N \leq 20$.
    \item \textbf{Subtask 4 [15 punti]:} $B=2$, $M,N \leq 500$ (come nella gara oii-2005, Milano).
    \item \textbf{Subtask 5 [30 punti]:} $B=3$, $M,N \leq 500$.
    \item \textbf{Subtask 6 [14 punti]:} $M,N \leq 500$.
    \item \textbf{Subtask 7 [4 punti]:} $B=2$, $M,N \leq 5000$.
    \item \textbf{Subtask 8 [5 punti]:} $B=3$, $M,N \leq 5000$.
    \item \textbf{Subtask 9 [7 punti]:} nessuna restrizionei, $M,N \leq 5000$.
  \end{itemize}
  


\end{document}


\end{document}
