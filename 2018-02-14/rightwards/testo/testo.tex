\documentclass[a4paper,11pt]{article}
\usepackage{nopageno} % visto che in questo caso abbiamo una pagina sola
\usepackage{lmodern}
\renewcommand*\familydefault{\sfdefault}
\usepackage{sfmath}
\usepackage[utf8]{inputenc}
\usepackage[T1]{fontenc}
\usepackage[italian]{babel}
\usepackage{indentfirst}
\usepackage{graphicx}
\usepackage{tikz}
\usepackage{wrapfig}
\newcommand*\circled[1]{\tikz[baseline=(char.base)]{
		\node[shape=circle,draw,inner sep=2pt] (char) {#1};}}
\usepackage{enumitem}
% \usepackage[group-separator={\,}]{siunitx}
\usepackage[left=2cm, right=2cm, bottom=2cm]{geometry}
\frenchspacing

\newcommand{\num}[1]{#1}

% Macro varie...
\newcommand{\file}[1]{\texttt{#1}}
\renewcommand{\arraystretch}{1.3}
\newcommand{\esempio}[2]{
\noindent\begin{minipage}{\textwidth}
\begin{tabular}{|p{11cm}|p{5cm}|}
	\hline
	\textbf{File \file{input.txt}} & \textbf{File \file{output.txt}}\\
	\hline
	\tt \small #1 &
	\tt \small #2 \\
	\hline
\end{tabular}
\end{minipage}
}

\newcommand{\sezionetesto}[1]{
    \section*{#1}
}

\newcommand{\gara}{Esame algoritmi 2018-02-14 VR}

%%%%% I seguenti campi verranno sovrascritti dall'\include{nomebreve} %%%%%
\newcommand{\nomebreve}{}
\newcommand{\titolo}{}

% Modificare a proprio piacimento:
\newcommand{\introduzione}{
%    \noindent{\Large \gara{}}
%    \vspace{0.5cm}
    \noindent{\Huge \textbf \titolo{}~(\texttt{\nomebreve{}})}
    \vspace{0.2cm}\\
}

\begin{document}

\renewcommand{\nomebreve}{rightwards}
\renewcommand{\titolo}{Procedi verso destra}

\introduzione{}

Data una matrice $val[1..M][1..N]$ di numeri naturali,
scegliete una qualsisi cella della sua prima colonna (la colonna pi\`u a sinistra) da cui partire
e quindi,
per $t=1,\ldots, N-1$,
eseguite un passo che vi porti dalla cella corrente $(x,t)$
ad una delle celle limotrofe della colonna successiva;
esse sono le seguenti:
\begin{itemize}
   \item la cella $(x,t+1)$ alla sua immediata destra;
   \item la cella $(x-1, t+1)$ ove esistente (ossia quando $x\geq 1$);
   \item la cella $(x+1, t+1)$ ove esistente (ossia quando $x\leq M$).
\end{itemize}     
Determinare la massima somma dei valori incontrati su un tale cammino.\\


\sezionetesto{Dati di input}
La prima riga del file \verb'input.txt' contiene, separati da spazio, i due interi positivi $M$ ed $N$, come nell'ordine. Essi rappresentano il numero di righe ed il numero di colonne della matrice $val$.
Le successive $M$ righe del file riportanto la matrice:
nella riga $i+1$ si trovano gli $N$ valori della riga $i$-esima della matrice, separati da spazi.
Si veda l'esempio.

\sezionetesto{Dati di output}
Nel file \verb'output.txt' si scriva un'unica riga contenente
un unico numero naturale: il massimo valore per la somma delle entries
incontrate su un cammino come descritto nel testo.\\


% Esempi
\sezionetesto{Esempio di input/output}
\esempio{
4 5

6 1 1 1 1

1 1 1 1 1

1 5 1 1 1

2 1 2 1 1
}{11}


\esempio{
4 6

6 1 1 1 4 1

3 1 1 1 1 1

1 5 1 1 1 2

1 1 2 1 1 1
}{15}

% Assunzioni
\sezionetesto{Assunzioni e note}
\begin{itemize}[nolistsep, noitemsep]
  \item $1 \le M,N \le 500$.
\end{itemize}
  
  \section*{Subtask}
  \begin{itemize}
    \item \textbf{Subtask 1 [0 punti]:} i due esempi del testo.
    \item \textbf{Subtask 2 [1 punti]:} tutti i valori della matrice uguali ad $1$.
    \item \textbf{Subtask 3 [2 punti]:} $M=1$.
    \item \textbf{Subtask 4 [4 punti]:} $N=1$.
    \item \textbf{Subtask 5 [8 punti]:} $N=2$.
    \item \textbf{Subtask 6 [16 punti]:} $M=2$.
    \item \textbf{Subtask 7 [32 punti]:} $M,N \leq 10$.
    \item \textbf{Subtask 8 [37 punti]:} nessuna restrizione.
  \end{itemize}
  


\end{document}
