\renewcommand{\nomebreve}{piastrelle1xk}
\renewcommand{\titolo}{Piastrelle da uno per kappa}

\introduzione{}

Dobbiamo piastrellare una griglia rettangolare di $k\times n$ celle quadrate
con delle piastrelle rettangolari da $1\times k$ quadretti. Le piastrelle possono essere ruotate ma devono ricoprire l'intera griglia senza sovrapporsi. 

Scrivere un programma che, dati $n$ e $k$, computi il numero di possibili soluzioni.\\


\sezionetesto{Dati di input}
La prima ed unica riga del file \verb'input.txt' contiene i due numeri naturali $n$ e $k$, in questo ordine, e separati da spazio.

\sezionetesto{Dati di output}
Nel file \verb'output.txt' si scriva solamente il numero di possibili tilings
di $n$ rettangoli $1\times k$ entro un rettangolo $k\times n$.\\


% Esempi
\sezionetesto{Esempio di input/output}
\esempio{
4 2
}{5}
\esempio{
4 3
}{3}
\esempio{
5 3
}{4}


% Assunzioni
\sezionetesto{Assunzioni e note}
\begin{itemize}[nolistsep, noitemsep]
  \item $0 \le n \le 1\,000\,000$, $2 \le k \le 1\,000\,000$.
  \item si garantisce che per tutte le istanze di nostro interesse il numero di soluzioni non ecceda il migliardo e possa quindi essere rappresentato senza problemi entro una variabile di tipo {\tt int}.  
\end{itemize}
  
  \section*{Subtask}
  \begin{itemize}
    \item \textbf{Subtask 1 [0 punti]:} gli esempi del testo.
    \item \textbf{Subtask 2 [10 punti]:} $n \leq 7$, $k=2$.
    \item \textbf{Subtask 2 [20 punti]:} $n \leq 30$, $k=2$.
    \item \textbf{Subtask 3 [20 punti]:} $n \leq 30$, $k=3$.
    \item \textbf{Subtask 4 [20 punti]:} $n \leq 30$.
    \item \textbf{Subtask 5 [30 punti]:} nessuna restrizione.
  \end{itemize}
  
